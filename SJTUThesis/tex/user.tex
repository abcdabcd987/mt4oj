%# -*- coding: utf-8-unix -*-
% !TEX program = xelatex
% !TEX root = ../thesis.tex

\chapter{User Models}

    The training process of a reinforcement learning involves frequent interaction
    between the reinforcement learning agent and users.
    Deploying an unreliable reinforcement learning algorithm in production is not a feasible option.
    On one hand, the agent is not well-trained, producing bad or misleading recommendations,
    which could jeopardize the education process.
    On the other hand, a single interaction between the agent and the student takes a long time.
    It takes hours if not days for a student to solve a problem,
    not to mention the even longer interval before attempting another problem.
    Hence, it could take tens of years to train the agent if not impossible at all.

    Therefore, we developed another approach.
    We firstly trained a \emph{user model} to capture students' mindsets using the existing submission records.
    Then, we programmed the reinforcement learning to interact with the user model.
    This real-time interaction enabled the successful training process.

    \todo{chapter overview}

\section{Problem Formulation}

    We formulate the user model as a binary classifier:
    given the user features, problem features, and the context features,
    the user model predicts whether the user is able to solve the problem,
    more precisely,
    it is the probability of the user able to solve the problem that we care about.

    The training and validation data is from the submission records of the Online Judge.
    We used the earliest 80\% records as the training set,
    and the most recent 20\% records is the validation set.
    For each submission record, the user features and the context features
    count statistics before the user made this submission.
    On the other side, we believe that the objective difficulty of each problem should not change over time.
    Hence, the more records are used, the better the generated problem features estimate the objective difficulty.
    Therefore, the problem features are the exactly the same for the same problem,
    generated using all of the submissions.

\section{Backgrounds}

    \subsection{Feature Scaling}

        The range of features in our feature set varies a lot.
        For example, $\verb|uf_ac_rate| \in [0,1]$, but $\verb|uf_avg_bytes|$ can be a few thousands.

        Feature scaling is a method to standardize the range of independent variables or features of data,
        making the learning process more robust.
        Another reason is that stochastic gradient decent converges much faster with feature scaling than without it.
        \cite{ioffe_batch_2015}

        We scaled each feature to $[0,1]$ in the following manner before feeding the input matrix $X$ to a user model.
        \[
        X_{:,j} \leftarrow \frac{X_{:,j}}{\max X_{:,j}}
        \]

        Notice that, not every user model is sensitive to feature scaling.
        For example, feature scaling has no effect on the boosted trees, because they deal with discrete values.

    \subsection{Metrics: Accuracy and AUC}

        To have a sense of the performance of a user model, we watch the following two metrics:
        classification accuracy and AUC.

        Classification accuracy, as the name suggests, is the ratio of
        the number of correct predictions to the total number of predictions made.
        A overfitted model tend to have a big gap between the train accuracy and the validation accuracy.

        Area under the receiver operating characteristic curve (\emph{AUC})
        is another most widely used metrics in classification tasks.
        There can be several equivalent interpretations of AUC \cite{flach_putting_2007}:

        \begin{itemize}
            \item The expectation that a uniformly drawn random positive is ranked before a uniformly drawn random negative.
            \item The expected proportion of positives ranked before a uniformly drawn random negative.
            \item The expected true positive rate if the ranking is split just before a uniformly drawn random negative.
            \item The expected proportion of negatives ranked after a uniformly drawn random positive.
            \item 1 – the expected false positive rate if the ranking is split just after a uniformly drawn random positive.
        \end{itemize}


\section{Logistic Regression}

    \todo{move from the background chapter}

\section{Factorization Machine}

    \subsection{Implementation}

        We used the \verb|fastFM|\cite{bayer_fastfm:_2016} Python library.
        We chose the stochastic gradient descent solver with the hyper-parameters shown in Table \ref{table:fm param}.

        \begin{table}[hpbt]
        \centering
        \begin{tabular}{lcl}
            \hline
            Hyper-parameter & Value & Description \\
            \hline
            \verb|n_iter|    & 100,000 & The number of iterations of individual samples. \\
            \verb|init_std|  & 0.1 & The standard deviation for the initialization of the parameters. \\
            \verb|rank|      & 2 & The rank of the factorization used for the second order interactions. \\
            \verb|l2_reg_w|  & 0 & L2 penalty weight for linear coefficients. \\
            \verb|l2_reg_V|  & 0 & L2 penalty weight for pairwise coefficients. \\
            \verb|l2_reg|    & 0 & L2 penalty weight for all coefficients. \\
            \verb|step_size| & 0.001 & Step size for the SGD solver. \\
            \hline
        \end{tabular}
        \caption{The hyper-parameters of the factorization machine user model}
        \label{table:fm param}
        \end{table}

    \subsection{Result}

        Table \ref{table:fm result} shows the validation result of the factorization machine user model.

        \begin{table}[hpbt]
        \centering
        \begin{tabular}{lll}
            \hline
            Feature Set & Accuracy & AUC \\
            \hline
            Basic    & \verb|0.6877472027750164| & \verb|0.6835856799533652| \\
            Extended & \verb|0.6936271917764758| & \verb|0.6580982070227229| \\
            \hline
        \end{tabular}
        \caption{The result of the factorization machine user model}
        \label{table:fm result}
        \end{table}

\section{Boosted Tree}

    \subsection{Implementation}

        We used the \verb|xgboost|\cite{Chen2016} Python library
        with the hyper-parameters shown in Table \ref{table:xgboost param}.

        \begin{table}[hpbt]
        \centering
        \begin{tabular}{lcl}
            \hline
            Hyper-parameter & Value & Description \\
            \hline
            \verb|booster| & gbtree & We use gradient boosting tree as the booster method. \\
            \verb|num_boost_round| & 10 & The number of boosting iterations. \\
            \verb|max_depth| & 7 & Maximum tree depth for base learners. \\
            \verb|learning_rate| & 0.3 & Boosting learning rate. \\
            \hline
        \end{tabular}
        \caption{The hyper-parameters of the boosted tree user model}
        \label{table:xgboost param}
        \end{table}

    \subsection{Result}

        Table \ref{table:fm result} shows the validation result of the boosted tree user model.

        \begin{table}[hpbt]
        \centering
        \begin{tabular}{lll}
            \hline
            Feature Set & Accuracy & AUC \\
            \hline
            Basic    & \verb|0.7029230736690708| & \verb|0.7104618544225273| \\
            Extended & \verb|0.7184056346369424| & \verb|0.7352554610590846| \\
            \hline
        \end{tabular}
        \caption{The result of the boosted tree user model}
        \label{table:xgboost result}
        \end{table}

\section{Recurrent Neural Networks}

    Table \ref{table:rnn result} shows the validation result of the boosted tree user model.

    \subsection{Implementation}

        We use the \verb|Keras|\cite{chollet2015keras} Python library to build the neural network.
        The neural network consists of a LSTM layer and a fully connected layer.
        In addition to features at the current time step,
        the input vector also contains features of the user's five previous submissions ($\verb|lookback| = 5$)
        and whether those submissions were accepted.
        The output dimension of the LSTM layer is 10.
        The fully connected layer only has a scalar output activated by the sigmoid function.
        We use the Adam optimizer\cite{kingma_adam:_2014}.

    \subsection{Result}

        \begin{table}[hpbt]
        \centering
        \begin{tabular}{lll}
            \hline
            Feature Set & Accuracy & AUC \\
            \hline
            Basic    & \verb|0.7163222572389433| & \verb|0.7232273437233644| \\
            Extended & \verb|0.7194526110958354| & \verb|0.7308123624422472| \\
            \hline
        \end{tabular}
        \caption{The result of the recurrent neural networks user model}
        \label{table:rnn result}
        \end{table}

\section{Counting Features}

    \todo{counting feature: count before vs. after}










