%# -*- coding: utf-8-unix -*-
% !TEX program = xelatex
% !TEX root = ../thesis.tex

\chapter{Related Works}
\label{chap:related}

\section{Machine Teaching}

    Machine teaching is not a brand new research topic.
    There has been a long history of related work dated back to 1990s.
    Recently, machine teaching started to draw researchers' attention again.
    Xiaojin Zhu's paper in Socratic dialogue style \cite{Zhu2015} stimulates critical thinking of
    what machine learning is and how it might help machines and humans.
    The paper also proposed several open problems and calls on studies of the research community.
    The longer version \cite{Zhu2018} of this paper explains things in more detail.

    As in any field, theories are the pillar of machine teaching.
    Although finding the optimal training set is in general hard,
    several works provide solutions to specific learners.
    \textcite{zhu2013machine} presents an approximate algorithm for finding the optimal teaching set
    for Bayesian learners which employ conjugate exponential family models.
    \textcite{xuezhou_zhang_optimal_2016} studies online learners, especially the perceptron.
    \textcite{liu2016teaching} presents the first known teaching dimension for ridge regression,
    support vector machines, and logistic regression.
    \textcite{zhu2017no} studies an interesting setting similar to the real-world classroom
    where the teacher must use the same training set to teach multiple learners.
    \textcite{ma2018teacher} proposes a method to trim down an independent identically distributed training set
    while making two kinds of simple learners learn even better.
    They also provide a mixed-integer nonlinear programming-based algorithm for general learners.

    Machine teaching applications in cognitive psychology and education show potential impacts on the real world.
    \textcite{khan2011humans} proposed a theoretical framework which assumes a teaching goal of
    minimizing the learner's expected generalization error at each iteration.
    The study extends the standard teaching dimension model and
    offers a theoretical justification for curriculum learning.
    \textcite{Patil2014} applies a machine teaching procedure
    to a cognitive model that is limited capacity as humans are.
    Based on the human study of a one-dimensional classification task they conducted,
    they found that the machine teacher recommends idealized training sets.

\section{Second Language Learning}

    Second language learning is a hot market where computer-aided teaching methods play an important role.
    One critical aspect of second language learning is to memorize the vocabulary.
    There are several commercial software that helps students with the vocabulary,
    such as Duolingo\cite{duolingo}, Shanbay\cite{shanbay}, SuperMemo\cite{supermemo}, and so on.
    They tackle this problem in part by using interaction data to estimate student proficiency and recommend content.
    Efficient language learning requires a deep understanding of human cognition.
    It also attracts the attention of researchers in varies of areas.

    In 1885, Ebbinghaus found that people tend to remember things more effectively
    if they use spaced repetition practice as opposed to the massive practice. \cite{ebbinghaus2013memory}
    This is called the spacing effect.
    Therefore, vocabulary software usually designs the learning plan according to \emph{spaced repetition models}.
    Central to the theory of memory is the Ebbinghaus model, also known as the \emph{forgetting curve},
    which approximates forgetting with an exponential curve \cite{bliss1993synaptic}.
    The probability of recalling an item is \cite{reddy2016unbounded}
    \[ Z \sim \mathrm{Bernoulli}\left( e^{-\theta\cdot\frac{D}{S}} \right) \]
    where $\theta$ is the item difficult,
    $D$ is the time elapsed since the item was last reviewed,
    $S$ is the student's memory strength for the item.

    \textcite{pimsleur1967memory} made an early attempt to put spaced repetition to practical use,
    whereby new vocabulary is introduced and then tested at exponentially increasing intervals,
    interspersed with the introduction or review of other vocabularies.
    However, this approach is limited since the schedule is pre-recorded and cannot adapt to the learner's actual ability.
    \textcite{leitner1972so} proposed a different spaced repetition algorithm intended for use with flashcards.
    It is more adaptive than Pimsleur's
    since the spacing intervals can increase or decrease depending on student performance.
    \textcite{settles2016trainable} proposed \emph{half-life regression} model,
    combined psychological theory with modern machine learning techniques.
    Instead of estimating the item difficulty and the student's memory strength using domain experience,
    they use the history of students' interactions with flashcards to fit a linear model:
    \[ Z \sim \mathrm{Bernoulli}\left( \exp\left( -\frac{D}{\exp(\bm{\theta}^T\bm{x})} \right)\right) \]
    where $\bm{x}$ is the input features and $\bm{\theta}$ is the parameters.

\section{Reinforcement Learning}

    Recent researches show that reinforcement learning
    is highly competitive to human players when playing games.
    \textcite{mnih_human-level_2015} presented the DQN algorithm and demonstrated that
    the DQN agent trained in end-to-end fashion is able to achieve a level comparable to
    professional human game testers across dozens of Atari games.
    DeepMind's AlphaGo\cite{silver2016mastering}
    has even the best professional human Go game player in the world \cite{silver2017mastering}.

    On the education field, researchers have made several attempts.
    \textcite{rafferty2016faster} formulated teaching as a
    partially observable Markov decision process planning problem.
    They presented approximate methods for finding optimal teaching actions,
    given the large state and action spaces that arise in teaching.
    \textcite{reddy_accelerating_2017} proposed a spaced repetition model
    that learns a policy that directly operates on raw observations of the study history
    using a reinforcement learning algorithm,
    showing that model-free scheduling is competitive against widely-used heuristics.
















