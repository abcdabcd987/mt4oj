%# -*- coding: utf-8-unix -*-

\chapter{Background}
\label{chap:background}

\section{Online Judge}
    
    \subsection{Programming Education}

        As we walk into a century surrounded by all kinds of computing devices,
        programming skill gains more and more attention.  % TODO: cite
        Universities used to teaching programming to merely computer science majored students,
        but more and more of them are opening the course to all.  % TODO: cite

        One part of programming education is mastering one programming language.
        Mechanisms of modern computers may be too complicated for non computer science majored students to learn,
        but thanks to the abstractions of processors, operating systems, and high-level programming languages,
        it becomes easier and easier to program computers to do what users want.
        The most popular high-level programming languages includes C++, Python, Java, and so on. % TODO: cite
        Instead of dealing with registers and memory addresses using processors' instruction set,
        high-level programming languages bring up concepts like variables, arrays, expressions, loops, functions,
        threads, processes, and other computer science abstractions.

        The other part of programming education is learning programmatic thinking, in other words,
        algorithms and data structures, especially for computer science majored students.
        There may be several solutions to the same problem, but taking different approaches costs differently.
        An $O(n\log n)$ algorithm better scales to a larger input comparing to an $O(n^2)$ algorithm.
        Programming also helps students cultivate thinking skills.
        There are mind sports specially focus on algorithms.
        For instance, \emph{ACM International Collegiate Programming Contest} (ACM-ICPC) is an annual
        competitive programming competition among the universities of world.
        In 2017, 49,935 students from 3,098 universities in 111 countries participated. % TODO: cite (wiki)
        Companies value this kind of thinking skills, as well.
        It is a common practice for companies to ask algorithm puzzles when interviewing candidate programmers.

        Higher level coursers in computer science, like Networking, Machine Learning, and so on,
        need students to be able to express their mind in code.
        Therefore, It is crucial for students to master programming at the very beginning of study,
        which puts challenges to entry-level courses, such as Programming Language, Data Structures, and so on.

        Like any other skills, both mastering one programming language and learning programmatic thinking
        require students to practice repeatedly.
        Important ways of these entry-level courses to help students master programming are assignments and exams.
        Problems in the assignments and exams are likely to be tasks asking students to write code.
        Graphics interfaces, keyboard and mouse inputs, video and audio outputs, networking connections,
        reading from and writing to disks, and so on, are common operations what programs in end-users' computers would have,
        but it would be too much burden for starters.
        Because these problems are for educational purpose only, the tasks are simplified from the real world ones.

        Descriptions of these problems are simple and neat. The tasks are idealized.
        Students do not need to deal with neither incorrect inputs nor malicious data.
        Problems are algorithmic in nature, thus there is no need to consider human-computer interaction.
        Memory is assumed to be large enough for the problems, therefore students can avoid disk operations
        and keep everything in memory.
        In a word, these assignments and exams ask students to write code that focus on the ``computing'' part of programming.

        After students finish the tasks, there need to be some ways to give them feedback, at least,
        tell students whether their solution is correct or not.
        Traditionally, grading programming solutions had no difference from grading calculus homework.
        And the emerge of online judges two decades ago enabled more efficient grading.
        
    \subsection{Manual Grading}

        Before the appearance of online judges, teachers needed to grade programming solutions manually.
        First, students handed in their solutions.
        Then, graders (either the professor himself/herself or the teaching assistants)
        would read the code and give marks based on their professional judgment.
        Manual grading is still a common practice nowadays in other subjects, such as mathematics and physics.
        Though widely used, its effectiveness is still affected by several non-controllable factors
        and has significant drawbacks. \cite{Kurnia2001}

        \subsubsection{Discouragement of Alternative Solutions}

            When the teachers are sketching a problem, they have a standard solution in mind.
            So when graders are grading students' solutions, they expect the code to be similar to
            the standard solution that they were told.
            However, it is likely that there are several alternative solutions to the same problem
            with the same complexity if not better.
            For instance, suppose an assignment can reduce to a minimum spanning tree problem
            and the author expects the Prim algorithm.  % TODO: cite Prim
            A student might turn in a solution with Kruskral algorithm which is also an algorithm that
            solves the minimum spanning tree problem.  % TODO: cite Kruskral
            In this case, if the grader is not aware of Kruskral algorithm, the grader might give a wrong verdict
            because Kruskral algorithm looks completely different from Prim algorithm, i.e. the standard solution.
            Notice that this is not the fault of the grader, because the grader might be a teaching assistant
            who is only expected to be familiar with the standard solution.
            Neither is this the fault of the problem author, because there can be any number of alternative solutions
            and the author cannot enumerate them all.
            Nor should we blame the student.
            In the contrary, students with alternative solutions should be encouraged.

        \subsubsection{Slow Grading}

            Grading a programming solution can takes a long time.
            Unlike a solution to mathematics homework in which students will explain their thinking process
            step by step in natural language,
            a solution to programming problem is code in some programming language which is designed
            for machines to compile, instead of letting humans to read.
            The grader have to read the code carefully in order to understand the solution.
            Then they need to look into details trying to find potential mistakes.
            Consequently, the grading process can be very slow.

        \subsubsection{Inconsistency}

            The same solution might have different scores if it is graded by different graders.
            For example, some graders know more than the standard solutions,
            and thus justify the alternative solutions.

            Even for the same grader to check the same solution at different times might lead to different results.
            For instance, at the beginning of grading, the grader might be very patient
            and carefully reason about each line of code.
            However, after hours of grading, the grader could feel tired and distracted
            and might want to finish grading as soon as possible.
            In this case, the grader might only check if the solution matches some patterns of the standard solution.

            Needless to say, the same solution is likely to have completely different verdicts
            depending on who the grader is and what status he or she is in.

        \subsubsection{Ignorance of Details Mistakes}

            Because of understanding students' solution costs graders lots of energy,
            subtle mistakes hidden in details often escape from being found.
            For example, supposing the standard solution is Floyd algorithm, % TODO: cite
            the grader might expect students' solutions to have two kinds of patterns:
            three nested loops and a dynamic programming equation.
            However, the solution might mixing up the permutation of the three loop variables,
            which makes the algorithm no longer correct.
            The grader might not be able to find such subtle an error.
            % TODO: correct & incorrect floyd code

        \subsubsection{More Manpower}

        \subsubsection{Long Feedback Time}

    \subsection{Automatic Grading}

\section{Factorization Machine}

\section{Boosting Tree}

\section{Recurrent Neural Network}

\section{Reinforcement Learning}

    \subsection{Asynchronous Advantage Actor Critic}
