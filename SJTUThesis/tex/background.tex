%# -*- coding: utf-8-unix -*-

\chapter{Background}
\label{chap:background}

\section{Online Judge}
    
    \subsection{Programming Education}

        As we walk into a century surrounded by all kinds of computing devices,
        programming skill gains more and more attention.  % TODO: cite
        Universities used to teaching programming to merely computer science majored students,
        but more and more of them are opening the course to all.  % TODO: cite

        One part of programming education is mastering one programming language.
        Mechanisms of modern computers may be too complicated for non computer science majored students to learn,
        but thanks to the abstractions of processors, operating systems, and high-level programming languages,
        it becomes easier and easier to program computers to do what users want.
        The most popular high-level programming languages includes C++, Python, Java, and so on. % TODO: cite
        Instead of dealing with registers and memory addresses using processors' instruction set,
        high-level programming languages bring up concepts like variables, arrays, expressions, loops, functions,
        threads, processes, and other computer science abstractions.

        The other part of programming education is learning programmatic thinking, in other words,
        algorithms and data structures, especially for computer science majored students.
        There may be several solutions to the same problem, but taking different approaches costs differently.
        An $O(n\log n)$ algorithm better scales to a larger input comparing to an $O(n^2)$ algorithm.
        Programming also helps students cultivate thinking skills.
        There are mind sports specially focus on algorithms.
        For instance, \emph{ACM International Collegiate Programming Contest} (ACM-ICPC) is an annual
        competitive programming competition among the universities of world.
        In 2017, 49,935 students from 3,098 universities in 111 countries participated. % TODO: cite (wiki)
        Companies value this kind of thinking skills, as well.
        It is a common practice for companies to ask algorithm puzzles when interviewing candidate programmers.

        Higher level coursers in computer science, like Networking, Machine Learning, and so on,
        need students to be able to express their mind in code.
        Therefore, It is crucial for students to master programming at the very beginning of study,
        which puts challenges to entry-level courses, such as Programming Language, Data Structures, and so on.

        Like any other skills, both mastering one programming language and learning programmatic thinking
        require students to practice repeatedly.
        Important ways of these entry-level courses to help students master programming are assignments and exams.
        Problems in the assignments and exams are likely to be tasks asking students to write code.
        Graphics interfaces, keyboard and mouse inputs, video and audio outputs, networking connections,
        reading from and writing to disks, and so on, are common operations what programs in end-users' computers would have,
        but it would be too much burden for starters.
        Because these problems are for educational purpose only, the tasks are simplified from the real world ones.

        Descriptions of these problems are simple and neat. The tasks are idealized.
        Students do not need to deal with neither incorrect inputs nor malicious data.
        Problems are algorithmic in nature, thus there is no need to consider human-computer interaction.
        Memory is assumed to be large enough for the problems, therefore students can avoid disk operations
        and keep everything in memory.
        In a word, these assignments and exams ask students to write code that focus on the ``computing'' part of programming.

        After students finish the tasks, there need to be some ways to give them feedback, at least,
        tell students whether their solution is correct or not.
        Traditionally, grading programming solutions had no difference from grading calculus homework.
        And the emerge of online judges decades ago enabled more efficient grading.
        
    \subsection{Manual Grading}

    \subsection{Automatic Grading}

\section{Factorization Machine}

\section{Boosting Tree}

\section{Recurrent Neural Network}

\section{Reinforcement Learning}

    \subsection{Asynchronous Advantage Actor Critic}
