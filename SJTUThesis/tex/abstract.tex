%# -*- coding: utf-8-unix -*-
% !TEX program = xelatex
% !TEX root = ../thesis.tex

\begin{abstract}

    编程教育吸引了全社会的关注,包括从小学生到计算机专业大学生的各个群体,也包括从商业公司的产品到国家层面的政策指导。
    在线评测系统的诞生给学生们带来了主动练习编程的独特机会。
    只需轻点鼠标,学生们就能方便地访问到在线评测系统上提供的成千上万道题目。
    利用在线评测系统,学生们在自主尝试解决课外的习题时,
    无须付出任何诸如聘请家庭教师之类的人力成本,就能瞬间得到对题目解法的高质量的反馈。
    然而,学生们如果缺乏有经验的老师中指导,常常会不知道应该去尝试在线评测系统上面的哪些题目。
    而有老师自身也是凭借经验挑选题目,缺乏统计数据的支撑。

    近些年的研究表明机器教学方法同样适用于人类学生。
    我们提出一个基于机器教学的智能在线评测系统,借此为学生们定制属于他们自己的训练题库,因材施教。
    我们使用增强学习算法来为学生们设计题库。
    为了解决增强学习算法的训练过程需要大量交互数据的问题,
    我们首先使用真实的在线评测系统的提交记录训练一个用户模型,对学生的水平进行建模,然后让增强学习算法与这个用户模型进行交互。
    实验数据表明,使用我们的用户模型提供的评测指标,我们的推荐算法表现优于基准算法以及人类教师。

\keywords{\large 编程教育 \quad 在线评测系统 \quad 增强学习 \quad 机器教学 \quad 机器学习 \quad 推荐系统}
\end{abstract}

\begin{englishabstract}

    Programming education has attract attention of the whole society,
    from primary school students to computer science majors,
    from products of commercial companies to nation-level policy guidance.
    The invention of online judges brought students a unique opportunity of practice programming proactively.
    Thousands of problems are easily accessible to students on online judges within a single click,
    where they can have consistent feedback instantly without any cost of manpower like private tutors
    whenever they attempt to solve an extracurricular problem.

    However, students without advice of experienced teachers often do not know which problems to do.
    Experienced teachers themselves pick problems in an ad-hoc manner,
    lacking the support of data and statistical analysis.

    Recent researches show that machine teaching techniques can also apply to human students.
    We propose to build an intelligent online judge system with machine teaching techniques
    so that it can help students to find their own optimal problem set.
    We use a reinforcement learning algorithm to design the problem set.
    To tackle with the requirement of the large number of interactive data
    during the training process of the reinforcement learning agent,
    we firstly fit a user model that captures students mindset using submission records from real-world online judges,
    then let the reinforcement learning agent interact with the user model.
    Experiment data shows that our algorithm outperforms baseline methods and human teachers,
    based on metrics of our user model.

\englishkeywords{\large Programming Education, Online Judge, Reinforcement Learning, Machine Teaching, Machine Learning, Recommendation System}
\end{englishabstract}

